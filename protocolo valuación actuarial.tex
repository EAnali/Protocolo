\documentclass[12pt,letterpaper,titlepage]{article}
\usepackage[spanish]{babel}
\usepackage[utf8x]{inputenc}
\usepackage{amsmath,amssymb}
\usepackage{graphicx}
\usepackage{amsthm}
\usepackage{latexsym}
\usepackage{cancel}
\usepackage{mathrsfs,amsfonts,mathptmx}

% diseño de página
\setlength{\parindent}{5ex}				% sangría
\usepackage[inner=1.6in,outer=1in,top=1in,bottom=1in]{geometry}
\usepackage{setspace}					% interlineado

\usepackage{color,longtable,pdfpages}
\usepackage{graphicx,hyperref}
\usepackage{url,breakurl}
\usepackage{skak}
\setcounter{secnumdepth}{3}
\setcounter{tocdepth}{4} 
\setlength\LTleft{0pt} \setlength\LTright{0pt} % parámetros para tablas largas

%\usepackage[text={5.8in,8.6in},centering]{geometry}
\renewcommand{\spanishoperators}{sen spec d}
\renewcommand{\baselinestretch}{1.6}
\makeatletter \decimalpoint
  \def\th@exercise{%
    \normalfont % body font
    \thm@headpunct{:}%
  }

\usepackage{apacite}
\makeatother
\usepackage{hyperref}
\hypersetup{% bookmarksnumbered,
  bookmarksopen,
  pdfpagelayout=OneColumn,
  pdfview=FitH,
  pdfstartview=FitH,
  pdfborder={0 0 0}}




% \theoremstyle{definition}
\renewcommand{\spanishrefname}{Bibliografía preliminar}

\title{Protocolo de trabajo de graduación:\\
La Valuación Actuarial de los Planes de Pensiones Ocupacionales}
\author{Br. Emilene Analí Romero Marroquín\\Carné: 2012-13113\\Dirección: 7a. Ave. "A" 9-29 Zona 11,\\Mixco, Guatemala\\Teléfono: 3335\,4941\\Asesores: Dr. Roberto Molina Cruz\\Licenciatura en Matemática Aplicada\\Universidad de San Carlos de Guatemala\\Número de páginas: \pageref{fin}}
\date{Guatemala, \today}

\begin{document}
\begin{titlepage}
\renewcommand{\thepage}{}
\pagestyle{empty}
\maketitle
\end{titlepage}\newpage
\setcounter{page}{2}
\tableofcontents
% % % % % % % % % % % % % % % % % % % % % % % % % % % % % % % % % % % % % % % %
\newpage
\nocite{*}
\section{Introducción}

En Guatemala los planes de pensiones ocupacionales, son beneficios que algunas empresas ofrecen a sus trabajadores como un complemento a los ya establecidos que los mismos obtienen del plan IVS (invalidez, vejez y supervivencia) que el Instituto Guatemalteco de Seguridad Social (IGSS) ofrece a sus asegurados. 

La estructuración de los planes de pensiones, varían según criterio de la empresa que lo administre así como también la metodología utilizada para las valuaciones acturiales de los mismos. Sin embargo, existen métodos de financiamiento recomendados a los planes de pensiones ocupacionales como también sistema financieros para los planes de pensiones sociales, los cuales también tienen su metodología específica para lo que son sus valuaciones actuariales.

Por ello se estudiará individualmente cada una de las formas de recaudación de fondos para los planes de pensiones, evidenciando de esta manera las similitudes y diferencias de un sistema a otro, como también las condiciones necesarias de un sistema de carácter social a las de un sistema de carácter privado.

Posteriormente se analizarán los métodos de valuación actuarial, mediante los cuales puede realizarse la misma, determinando el enfoque de cada uno y su área de aplicación, identificando cuál es el recomendado para planes de carácter social o privado.




% % % % % % % % % % % % % % % % % % % % % % % % % % % % % % % % % % % % % % % %
\newpage

\section{Justificación}

La \textit{Licenciatura en Matemática Aplicada} de la \textit{Universidad de San Carlos de Guatemala} (USAC) tiene como fin primordial contribuir al desarrollo de la ciencia del país, a través de la formación de profesionales con un enfoque curricular y un modelo pedagógico que le capacitan para desempeñarse con éxito en los distintos sectores del campo laboral, de esta manera reduciendo la dependencia de científica de los países desarrollados. Sin embargo la contribución que de esta se tiene al campo actuarial es mínimo, por lo que es considerable realizar un estudio con enfoque a dicha disciplina contribuyendo de esta forma a la bibliografía en la cual se tenga en cuenta las necesidades  del país, enfocandose principalmente en los beneficios que un plan de pensiones puede ofrecerse al trabajador. \bigskip

En Guatemala los planes de pensiones ocupacionales, en cuanto al análisis de su situación financiera, han dado mayor importancia al método del análisis del flujo de efectivo, mientras que las normas internacionales dictan que estos planes deben ser evaluados principalmente por el nivel de financiamiento de su reserva matemática.  \bigskip

El estudio de los aspectos matemáticos de ambos métodos presentados en este trabajo de graduación, determinará la conveniencia de su aplicación en los planes de pensiones ocupacionales. Dejando de esta forma, un precedente que sirva de base en el ámbito guatemalteco, para análisis posteriores de la situación financiera de diversos planes de pensiones ocupacionales.



% % % % % % % % % % % % % % % % % % % % % % % % % % % % % % % % % % % % % % % %
\newpage

\section{Marco teórico}

La \textbf{valuacion actuarial de los planes de pensiones} puede varíar según sean estos de una institución pública o privada. Entiéndase por \textbf{valuación actuarial} a la serie de pasos mediante la cual el actuario analiza la situación financiera de un plan de pensiones específico. Dicha valuación puede realizarse a través el cálculo de flujo de efectivo o de la reserva matemática\footnote{Es en la que se hace una separación de la prima recibida por el asegurador del asegurado, para cubrir futuras contingencias que están contempladas en el mutuo acuerdo.} del plan. Por otro lado, es preciso conocer la teoría detrás de este peculiar proceso. Comenzaremos definiendo que es un plan de pensiones, siendo el objetivo primordial de dicho trabajo los planes de pensiones ocupacionales. 

Durante el siglo XIX con el surgimiento de la Revolución Industrial, en los países conocidos como desarrollados, en Europa o Estados Unidos por ejemplo,  se promovió la creación de una institución que velara por la seguridad económica del trabajador, en caso de este encontrarse en una situación desventajosa para poder proveerse de un modo para obtener ingresos, siendo esta la \textbf{seguridad social}. Entre los beneficios que la seguridad social se encuentran los \textbf{planes de pensiones}, los cuales son convenios institucionales que protegen en la vejez, la invalidez y en el caso de fallecimiento, del proveedor de sustento del hogar, a los dependientes quienes sufren la pérdida. Con el tiempo distintintas instituiones privadas fueron implementando un programa similar para sus trabajadores o para un grupo en específico, como lo que proveen ciertos colegios de profesionales. Los planes de pensiones se denominan \textbf{sociales} cuando estos son administrados por el seguro social y \textbf{ocupacionales} cuando su administración corresponde a una entidad privada.  En Guatemala, los planes de pensiones ocupacionales son otorgados por algunas empresas a sus trabajadores, como complemento de los beneficios brindados por el Insituto Guatemalteco de Seguridad Social (IGSS). 

Los planes de pensiones pueden clasificarse mediante el régimen al que estén asociados. Estos pueden ser de \textit{prestación definido,  cotización definido} o \textit{mixtos}. Un régimen se le llama de \textbf{prestación definido (PD)} cuando el monto de prestación se define mediante una fórmula, la cual es independiente del monto de cotización que el cotizante aporta a lo largo de su carrera laboral. Los planes de \textbf{contribución definida (CD)} se caracterizan por que el monto de la pensión es calculado en el momento que el miembro del Plan califica a la pensión, en base en lo que contribuyo en su vida laboral. 

Estas contribuciones pueden ser capitalizadas en forma individual, para  cada miembro del Plan, o en forma colectiva para todos los miembros. Así un caso particular de este tipo de planes son los planes de \textbf{capitalización individual}.

En un plan de pensiones las fuentes de financiamiento son mediante contribuciones, aportadas por los individuos participantes del plan, así como también pueden ser en conjunto por los trabajadores y el empleador, o en el caso del sistema público puede incluir subsidios gubernamentales ó impuestos destinados al seguro social. Entre los parámetros básicos demográficos y económicos asociados a un régimen de pensiones se encuentran los siguientes:
 
\begin{itemize}
	\item [$\bullet$] La fuerza de interés $\delta$.
 	\item [$\bullet$] La fuerza de crecimiento de nuevos afiliados $\rho$.
 	\item [$\bullet$] La fuerza de incremento del salario asegurado $\gamma$.
 	\item [$\bullet$] La fuerza de la indexación de pensiones $\beta$.
 	\item [$\bullet$] La fuerza de mortalidad, invalidez y otros decrementos (según la edad) $\mu_{x}^{d}, \mu_{x}^{t}$, sucesivamente.
 	\item [$\bullet$] La fuerza de inflación $\theta$.
 \end{itemize}
 
  
 Por otro lado, en lo general se tomara en cuenta que el número de nuevos afiliados en un intervalo $(0, dt)$ es \textit{dt}, por lo que el número de nuevos afiliados en el intervalo $(t, t+dt)$ es $e^{\rho t}dt$. Analogamente, si el nivel general al inicio del régimen se toma como una unidad monetaria, el nivel de salarios al tiempo \textit{t} deberá ser $e^{\gamma t}$. La unidad de pensión deberá crecer a $e^{\beta t}$ en \textit{t} años \cite{778}. Las dos funciones, ambas continuas y diferenciables, fundamentales que describen el desarrollo demográfico de un régimen de pensiones son:

\begin{itemize}
	\item [$\bullet$] La función de la población activa $A(t)$.
	\item [$\bullet$] La función de la población retirada $R(t)$.
\end{itemize}
 
 Las dos funciones, ambas continuas y diferenciables, que caracterizan el desarrollo financiero de un régimen de pensiones son:
\textit{ \begin{itemize}
 	\item [$\bullet$] La función de gastos $G(t)$.
 	\item [$\bullet$] La función del salario asegurado $S(t)$.
 \end{itemize}}
 
 Donde la función $G(t)$, se considera como una función relacionada al gasto únicamente de los beneficios. El total de los gastos de beneficios y la cuenta de salario total asegurado en el intervalo $(z, z+dz)$ serán entonces dados por $B(z)dz$ y $S(z)dz$. Previo a la discusión de los sistemas de financiamiento, es necesario establecer dos funciones:
 
 \begin{itemize}
 	\item [$\bullet$] La función del monto de contribución $C(t)$, la cual está determinada según el sistema de financiamiento.
 	\item [$\bullet$] La función de reserva $V(t)$, que representa el exceso de la entrada del flujo dinero de la salida.
 \end{itemize}

Estas últimas funciones están conectadas con las funciones $B(t)$ y $S(t)$ por la ecuación diferencial fundamental \cite[p. 15]{778}:

\begin{equation}\label{dV}
dV(t)=V(t)\delta dt+C(t)S(t)dt-B(t)dt.
\end{equation}

Es decir, el cambio en la reserva en un pequeño período de tiempo es igual a la ganacia obtenida al invertir la suma de la reserva más la diferencia entre ingreso de las contribuciones y el gasto asociado al pago de beneficios. Donde integrando la ecuación anterior respecto del tiempo en el intervalo de $(n,m)$, tomando $n=0$, se obtiene la siguiente:

\begin{equation}\label{Vt}
V(m)=V(n)e^{\delta(m)}+\int_{0}^m[C(t)S(t)-B(t)]e^{\delta(m-t)}dt.
\end{equation}

De la cual, dadas las condiciones ya sea para la función $V(t)$ o $C(t)$, se obtienen las ecuación que caracterizan a cada uno de los diferentes tipos de financiamiento de un plan. Por ejemplo, se tiene el \textit{sistema de reparto, PAYG} por sus siglas en inglés, el cual se caracteriza por su condición inicial para la reserva, en donde $V(t)=0$, reduciendo la ecuación~(\ref{Vt}) a:

\begin{equation}\label{payg}
C(t)=\frac{B(t)}{S(t)}.
\end{equation}

En donde el financiamiento de dicho sistema consiste en que los egresos totales se obtiene de los ingresos totales. También se tiene el \textit{sistema de prima media general, GAP} por sus siglas en inglés, el cual consiste de una contribución constante indefinidamente, por lo que para un tiempo $t=m$, $C(t)=C$, con $C$ constante, reduciendo~(\ref{Vt}) a:

\begin{equation}\label{gap}
C=\dfrac{\int_{m}^{\infty}B(t)e^{-\delta t}dt-V(m)e^{-\delta m}}{\int^{\infty}_{m}S(t)e^{-\delta t}dt}.
\end{equation}

A diferencia del sistema de prima media general, se tiene \textit{el sistema de prima escalonada}, el cual se basa en una sucesión de períodos, en los cuales el monto de contribución se determina mediante el análisis del balance de ingresos y egresos de acuerdo a un período limitado de años a estudiar $(n_{0}, n_{1})$, aplicandose a un período más corto $(n_{0}, n_{1}')$ durante el cual la reserva crece de forma continua hasta alcanzar su máximo en $t=n_{1}'$, posteriormete se calcula un aumento al monto de contribución para otro período, de la misma forma sucesivamente.

Los sistemas de financiamiento, anteriormente mencionados son comunmente los utilizados para un plan de pensiones de carácter social. Según \cite{778} los fondos para un plan de pensiones de carácter ocupacional pueden obtenerse por distintos métodos, para los cuales será necesario definir funciones adicionales:

\begin{itemize}
	\item [$\bullet$] La función del monto de contribución según la edad $K(x)$.
	\item [$\bullet$] La función de reserva por unidad de salario a pagar $F(x)$.
\end{itemize}

También se denotará la edad de ingreso al plan por $b$ y la edad de retiro por $r$. Entre los financiamientos para los planes de pensiones ocuapacionales, se encuentran los \textbf{métodos de costo individual}, los cuales determinan el equilibrio financiero de los nuevos cotizantes como primer punto, considerando posteriormente, los ajustes necesarios para lograr el equilibrio del fondo destinado a la población inicial, por simplicidad se asume que todos los nuevos cotizantes ingresan al plan a una misma edad $b$ y se retiran a una edad $r$. Los \textbf{métodos de costos de beneficios acumulados}, se caracterizan por el suministro de una porción de la pensión final a beneficio de sus cotizantes en un intervalo de tiempo, también se pueden mencionar los \textbf{métodos de costo según edad de primera cotización}, los cuales buscan establecer el monto de contribución en función de la edad de ingreso al plan, en cuyo caso $K(x)=K(b)$ para toda edad $x$.


La \textbf{valuación actuarial} de un plan de pensiones, la cual se realiza para determinar la situación financiera, puede realizarse mediante el análisis del \textbf{flujo de efectivo} o del \textbf{financiamiento de su reserva matemática}.

El primero se realiza mediante el \textbf{método de proyección demográfica-financiera}, el cual consiste en proyectar la población asegurada, la cual es clasificada según categoría, sexo y edad; para posteriormente hacer proyecciones financieras en base a la demografía y de esta manera examinar la suficiencia de los convenios financieros existentes. Mientras que el segundo se realiza mediante la \textbf{técnica del valor presente}, el cual toma en consideración una cohorte de la población afiliada al plan en un tiempo específico y calcula el valor presente estimado de los salarios asegurados futuros y las pensiones de beneficios pagaderos a los miembros de dicha cohorte y sus beneficiarios.

% % % % % % % % % % % % % % % % % % % % % % % % % % % % % % % % % % % % % % % %
\newpage

\section{Objetivos}


\subsection{Objetivo general}

\begin{itemize}
	\item Determinar los beneficios de evaluar la situación financiera de un plan de pensiones ocupacional, estimando el nivel del financiamiento de su reserva matemática contra el análisis del flujo de efectivo.
\end{itemize}


\subsection{Objetivos específicos}

\begin{enumerate}
	\item Identificar los principales beneficios que un plan de pensiones ocupacional puede brindar a sus asegurados.
	\item Estudiar la estructuración de los elementos de la matemática actuarial necesarios para el manejo de un plan de pensiones ocupacional.
	\item Estudiar los aspectos matemáticos de los métodos de valuación actuarial para su aplicación en un plan de pensiones ocupacional.
	\item Identificar cual de los métodos de valuación actuarial es el más conveniente en un plan de pensiones ocupacional. 
\end{enumerate}

% % % % % % % % % % % % % % % % % % % % % % % % % % % % % % % % % % % % % % % %
\newpage

\section{Metodología}

La metodología a utilizar en el desarrollo de este trabajo de graduación, se llevará a cabo de la siguiente manera:

\begin{itemize}
	\item Se consultará, en primera instancia, diversos documentos de acceso público de la Organización Internacional del Trabajo, incluídas en la bibliografía preliminar,  en los cuales establecen las normas y convenios para la estructuración de los distintos planes de pensiones, además de una variedad de textos sobre matemática actuarial.
	
	\item Se hará uso del \textit{análisis matemático} para evaluar los distintos métodos de evaluación y poder de esta manera determinar su utilidad.
	 
	\item Se utilizará el \textit{método analítico} para determinar las ventajas y desventajas de los métodos de valuación de la situación financiera de los planes ocupacionales.
	
	\item Por último se hará una compilación de los resultados y de esta manera concluir que es lo más recomendable para los planes ocupacionales, en cuanto al análisis de su situación financiera.
\end{itemize}


\newpage

\section{Índice preliminar}
El índice preliminar del trabajo de graduación es el siguiente:
\begin{align*}
&\text{Índice de Ilustraciones}\\
&\text{Listado de Símbolos}\\
&\text{Glosario}\\
&\text{Introducción}\\
&\text{1. Planes de Pensiones}\\
&\qquad\text{1.1. Aspecto Histórico }\\
&\qquad\text{1.2. Beneficios del Contribuyente }\\
&\qquad\qquad\text{1.2.1. Beneficio de Vejez}\\
&\qquad\qquad\text{1.2.2. Beneficio de Supervivencia}\\
&\qquad\qquad\text{1.2.3. Beneficio de Invalidez}\\
&\qquad\text{1.3. Regímenes de los Planes de Pensiones}\\
&\qquad\qquad\text{1.3.1. Régimen de Prestación Definida}\\
&\qquad\qquad\text{1.3.2. Régimen de Cotización Definida}\\
&\qquad\qquad\text{1.3.3. Régimen Mixto}\\
&\text{2. Matemática Actuarial de los Planes de Pensiones}\\
&\qquad\text{2.1. Funciones Básicas para el Financiamiento de los Planes de Pensiones}\\
&\qquad\text{2.2. Sistemas de Financiamiento de los Planes de Pensiones}\\
&\qquad\qquad\text{2.2.1. Sistema de Reparto}\\
&\qquad\qquad\text{2.2.2. Sistema de Prima Escalonada}\\
&\qquad\qquad\text{2.2.3. Sistema de Prima Media General}\\
\end{align*}

\newpage
\begin{align*}
&\text{3. Métodos de Valuación Actuarial}\\
&\qquad\text{3.1. El Análisis del Flujo de Efectivo}\\
&\qquad\text{3.2. El Financiamiento de su Reserva Matemática}\\
&\text{4. Aplicación de los Métodos de Valuación Actuarial en los Planes de Pensiones Ocupacionales}\\
&\text{Recomendaciones}\\
&\text{Conclusiones}\\
&\text{Bibliografía} 
\end{align*}

% % % % % % % % % % % % % % % % % % % % % % % % % % % % % % % % % % % % % % % %
\newpage

\section{}
\bibliographystyle{apacite}
\bibliography{bibliotesis}
\label{fin}
\end{document} 
