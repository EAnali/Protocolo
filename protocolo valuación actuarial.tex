\documentclass[12pt,letterpaper,titlepage]{article}
\usepackage[spanish]{babel}
\usepackage[utf8x]{inputenc}
\usepackage{amsmath,amssymb}
\usepackage{graphicx}
\usepackage{amsthm}
\usepackage{latexsym}
\usepackage{cancel}
\usepackage{mathrsfs,amsfonts,mathptmx}

% diseño de página
\setlength{\parindent}{5ex}				% sangría
\usepackage[inner=1.6in,outer=1in,top=1in,bottom=1in]{geometry}
\usepackage{setspace}					% interlineado

\usepackage{color,longtable,pdfpages}
\usepackage{graphicx,hyperref}
\usepackage{url,breakurl}
\usepackage{skak}
\setcounter{secnumdepth}{3}
\setcounter{tocdepth}{4} 
\setlength\LTleft{0pt} \setlength\LTright{0pt} % parámetros para tablas largas

%\usepackage[text={5.8in,8.6in},centering]{geometry}
\renewcommand{\spanishoperators}{sen spec d}
\renewcommand{\baselinestretch}{1.6}
\makeatletter \decimalpoint
  \def\th@exercise{%
    \normalfont % body font
    \thm@headpunct{:}%
  }

\usepackage{apacite}
\makeatother
\usepackage{hyperref}
\hypersetup{% bookmarksnumbered,
  bookmarksopen,
  pdfpagelayout=OneColumn,
  pdfview=FitH,
  pdfstartview=FitH,
  pdfborder={0 0 0}}




% \theoremstyle{definition}
\renewcommand{\spanishrefname}{Bibliografía preliminar}

\title{Protocolo de trabajo de graduación:\\
La Valuación Actuarial de los Planes de Pensiones Ocupacionales}
\author{Br. Emilene Analí Romero Marroquín\\Carné: 2012-13113\\Dirección: 7a. Ave. "A" 9-29 Zona 11,\\Mixco, Guatemala\\Teléfono: 3335\,4941\\Asesores: Dr. Roberto Molina Cruz\\Licenciatura en Matemática Aplicada\\Universidad de San Carlos de Guatemala\\Número de páginas: \pageref{fin}}
\date{Guatemala, \today}

\begin{document}
%\begin{titlepage}
%\renewcommand{\thepage}{}
%\pagestyle{empty}
\maketitle
%\end{titlepage}\newpage
\setcounter{page}{2}
\tableofcontents
% % % % % % % % % % % % % % % % % % % % % % % % % % % % % % % % % % % % % % % %
\newpage
\nocite{*}
\section{Introducción}





% % % % % % % % % % % % % % % % % % % % % % % % % % % % % % % % % % % % % % % %
\newpage

\section{Justificación}

La \textit{Licenciatura en Matemática Aplicada} de la \textit{Universidad de San Carlos de Guatemala} (USAC) tiene como fin primordial contribuir al desarrollo de la ciencia del país, a través de la formación de profesionales con un enfoque curricular y un modelo pedagógico que le capacitan para desempeñarse con éxito en los distintos sectores del campo laboral, de esta manera reduciendo la dependencia de científica de los países desarrollados.\bigskip

En Guatemala los planes de pensiones ocupacionales, en cuanto al análisis de su situación financiera, han dado mayor importancia al método del análisis del flujo de efectivo, mientras que las normas internacionales dictan que estos planes deben ser evaluados principalmente por el nivel de financiamiento de su reserva matemática.  \bigskip

El estudio de los aspectos matemáticos de ambos métodos presentados mediante este trabajo de graduación, determinará la conveniencia de su aplicación en los planes de pensiones ocupacionales. Dejando de esta forma, un presedente que sirva de base en el ámbito guatemalteco, para análisis posteriores de la situación financiera de diversos planes de pensiones ocupacionales.



% % % % % % % % % % % % % % % % % % % % % % % % % % % % % % % % % % % % % % % %
\newpage

\section{Marco teórico}

La \textbf{valuacion actuarial de los planes de pensiones} puede varíar según sean estos de una instución pública o privada. Entiéndase por \textbf{valuación actuarial} a la serie de pasos mediante la cual el actuario analiza la situación financiera de un plan de pensiones específico. Dicha valuación puede realizarse mediante el cálculo de flujo de efectivo o de la reserva matemática del plan. Por otro lado, es preciso conocer la teoría detrás de este peculiar proceso. Comenzaremos definiendo que es un plan de pensiones, siendo el objetivo primordial de dicho trabajo los planes de pensiones ocuapacionales. 

Los \textbf{planes de pensiones} tienen sus orígenes con la creación de la seguridad social. Durante el siglo XIX con el surgimiento de la Revolución Industrial, en los países conocidos como desarrollados, en Europa o Estados Unidos por ejemplo,  se promovió la creación de una institución que velara por la seguridad económica del trabajador, en caso de este encontrarse en una situación desventajosa para poder proveerse de un modo para obtener ingresos, siendo esta la \textbf{seguridad social}. Entre los beneficios que la seguridad social se encuentra lo que son los \textit{planes de pensiones}, los cuales son "arrangements" institucionales que protegen en la vejez, la invalidez y en el caso de fallecimiento, del proveedor de sustento del hogar, a los dependientes quienes sufren la pérdida. Con el tiempo distintintas instituiones privadas fueron implementando un programa similar para sus trabajadores o para un grupo en específico, como lo que proveen ciertos colegios de profesionales. Los planes de pensiones se denominan \textbf{sociales} cuando estos son administrados por el seguro social y \textbf{ocupacionales} cuando su administración corresponde a una entidad privada.

Los planes de pensiones pueden clasificarse mediante el régimen al que estén asociados. Estos pueden ser de \textit{prestación definido,  cotización definido} o \textit{mixtos}. Un régimen se le llama de \textbf{prestación definido (PD)} cuando el monto de prestación se define mediante una fórmula, la cual es independiente del monto de cotización que el cotizante aporta a lo largo de su carrera laboral. Por otro lado en un régimen de \textbf{cotización definido} una cuenta individual es creada a nombre de cada uno de los miembros del plan, en la cual las cotizaciones del miembro son registradas \cite{778}, incluyendo la acumulación de los intereses para posteriormente realizar el cálculo de la prestación en base a lo acumulado.


Entiendase por \textbf{planes de pensiones ocupacionales}, a los beneficios que las empresas ofrecen a sus trabajadores como un complemento del brindado por la seguridad social del país. 




%Los \textit{métodos de valuación actuarial} para determinar la situación financiera de un plan de pensiones son los siguientes: \textbf{}

% % % % % % % % % % % % % % % % % % % % % % % % % % % % % % % % % % % % % % % %
\newpage

\section{Objetivos}


\subsection{Objetivo general}

\begin{itemize}
	\item Determinar los beneficios de evaluar la situación financiera de un plan de pensiones ocupacional, estimando el nivel del financiamiento de su reserva matemática contra el flujo de efectivo.
\end{itemize}


\subsection{Objetivos específicos}

\begin{enumerate}
	\item Identificar los principales beneficios de los planes ocupacionales.
	\item Identificar los elementos de la matemática actuarial necesarios para el cálculo de pensiones.
	\item Revisar los aspectos matemáticos de los métodos de valuación actuarial.
	\item Comprobar la conveniencia de la aplicación de los métodos de valuación en los planes de pensiones ocupacionales. 
\end{enumerate}

% % % % % % % % % % % % % % % % % % % % % % % % % % % % % % % % % % % % % % % %
\newpage

\section{Metodología}

La metodología a utilizar en el desarrollo de este trabajo de graduación, se llevará a cabo de la siguiente manera:

\begin{itemize}
	\item Se consultará, en primera instancia, diversos documentos de acceso público de la Organización Internacional del Trabajo, incluídas en la bibliografía preliminar,  en los cuales establecen las normas y convenios para la estructuración de los distintos planes de pensiones, además de una variedad de textos sobre matemática actuarial.
	
	\item Se hará uso del \textit{análisis matemático} para evaluar los distintos métodos de evaluación y poder de esta manera determinar su utilidad.
	 
	\item Se utilizará el \textit{método analítico} para determinar las ventajas y desventajas de los métodos de valuación de la situación financiera de los planes ocupacionales.
	
	\item Por último se hará una compilación de los resultados y de eta manera concluir que es lo más recomendable para los planes ocupacionales, en cuanto al análisis de su situación financiera.
\end{itemize}

% % % % % % % % % % % % % % % % % % % % % % % % % % % % % % % % % % % % % % % %
\newpage

\section{Cronograma}
Se tabulan las actividades previstas en el desarrollo del trabajo de graduación.
% El cronograma es un cuadro semejante al utilizado en los anteproyectos FACYT,
% meses en las etiquetas de las columnas y actividades con etiquetas de las filas, 
% se sombrea (marca) la celda correspondiente.

\vspace{1cm}

\begin{tabular}{|c|c|c|}
  \hline
  % after \\: \hline or \cline{col1-col2} \cline{col3-col4} ...
   \textbf{No.}  &  \textbf{Actividad a realizar} &  \textbf{Duración }\\ \hline
  1  & Elaboración del protocolo & 2 semanas \\
  2  & Búsqueda de bibliografía & 3 semanas \\
  3  & Capítulo 1: Planes de Pensiones  & 5 semanas \\
  4  & Capítulo 2: Matemática Actuarial de los Planes de\\
  & Pensiones & 6 semanas \\
  5  & Capítulo 3: Tipos de Métodos de Valuación & 5 semanas \\
  6  & Capítulo 4: Aplicación de los Métodos de Valuación en los Planes de Pensiones & 6 semanas \\
  7  & Escritura de informe final & 1 semana \\
  8  & Revisión del asesor de tesis & 1 semana \\
  9 & Revisión del revisor de Escuela de Ciencias & 1-2 semanas \\
  10 & Revisión departamento de Lingüística & 1-2 semanas \\
  11 & Trámites finales & 1 semana  \\
  12 & Impresión informe final & 2-3 días \\
  13 & Solicitud Examen Público & 1 semana \\
  \hline
\end{tabular}

% % % % % % % % % % % % % % % % % % % % % % % % % % % % % % % % % % % % % % % %
\newpage

\section{Índice preliminar}
El índice preliminar del trabajo de graduación es el siguiente:
\begin{align*}
&\text{Índice de Ilustraciones}\\
&\text{Listado de Símbolos}\\
&\text{Glosario}\\
&\text{Introducción}\\
&\text{1. Planes de Pensiones}\\
&\qquad\text{1.1. Aspecto Histórico }\\
&\qquad\text{1.2. Beneficios del Contribuyente }\\
&\qquad\qquad\text{1.2.1. Beneficio de Vejez}\\
&\qquad\qquad\text{1.2.2. Beneficio de Supervivencia}\\
&\qquad\qquad\text{1.2.3. Beneficio de Invalidez}\\
&\qquad\text{1.3. Regímenes de los Planes de Pensiones}\\
&\qquad\qquad\text{1.3.1. Régimen de Prestación Definida}\\
&\qquad\qquad\text{1.3.2. Régimen de Cotización Definida}\\
&\qquad\qquad\text{1.3.3. Régimen Mixto}\\
&\text{2. Matemática Actuarial de los Planes de Pensiones}\\
&\qquad\text{2.1. Funciones Básicas para el Financiamiento de los Planes de Pensiones}\\
&\qquad\text{2.2. Sistemas de Financiamiento de los Planes de Pensiones}\\
&\qquad\qquad\text{2.2.1. Sistema de Reparto}\\
&\qquad\qquad\text{2.2.2. Sistema de Prima Escalonada}\\
&\qquad\qquad\text{2.2.3. Sistema de Prima Media General}\\
\end{align*}

\newpage
\begin{align*}
&\text{3. Métodos de Valuación Actuarial}\\
&\qquad\text{3.1. El Análisis del Flujo de Efectivo}\\
&\qquad\text{3.2. El Financiamiento de su Reserva Matemática}\\
&\text{4. Aplicación de los Métodos de Valuación Actuarial en los Planes de Pensiones Ocupacionales}\\
&\text{Recomendaciones}\\
&\text{Conclusiones}\\
&\text{Bibliografía} 
\end{align*}

% % % % % % % % % % % % % % % % % % % % % % % % % % % % % % % % % % % % % % % %
\newpage

\section{Bibliografía preliminar}
\bibliographystyle{apacite}
\bibliography{bibliotesis}
\label{fin}
\end{document} 
