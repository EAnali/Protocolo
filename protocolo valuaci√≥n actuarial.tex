%% Correcciones y sugerencias por William Gutiérrez
%% para no perder la costumbre :-)
%% Debe incluir al Dr. Sergio López, asesor principal
\documentclass[12pt,letterpaper,titlepage]{article}
\usepackage[spanish]{babel}
\usepackage[utf8x]{inputenc}
\usepackage{amsmath,amssymb}
\usepackage{graphicx}
\usepackage{amsthm}
\usepackage{latexsym}
\usepackage{cancel}
\usepackage{mathrsfs,amsfonts,mathptmx}
\usepackage[text={5.8in,8.6in},centering]{geometry}
\renewcommand{\spanishoperators}{sen spec d}
\renewcommand{\baselinestretch}{1.6}
\makeatletter \decimalpoint
  \def\th@exercise{%
    \normalfont % body font
    \thm@headpunct{:}%
  }

\makeatother
\usepackage{hyperref}
\hypersetup{% bookmarksnumbered,
  bookmarksopen,
  pdfpagelayout=OneColumn,
  pdfview=FitH,
  pdfstartview=FitH,
  pdfborder={0 0 0}}




% \theoremstyle{definition}
\renewcommand{\spanishrefname}{Bibliografía preliminar}

\title{Protocolo de trabajo de graduación:\\
La Valuación de los Planes de Pensiones Ocupacionales o Privados}
\author{Br. Emilene Analí Romero Marroquín\\Carné: 2012-13113\\Dirección: 7a. Ave. "A" 9-29 Zona 11,\\Mixco, Guatemala\\Teléfono: 3335\,4941\\Asesores: Dr. Roberto Molina Cruz\\Licenciatura en Matemática Aplicada\\Universidad de San Carlos de Guatemala\\Número de páginas: \pageref{fin}}
\date{Guatemala, \today}

\begin{document}
%\begin{titlepage}
%\renewcommand{\thepage}{}
%\pagestyle{empty}
\maketitle
%\end{titlepage}\newpage
\setcounter{page}{2}
\tableofcontents

% % % % % % % % % % % % % % % % % % % % % % % % % % % % % % % % % % % % % % % %
\newpage

\section{Introducción}

%Los planes de pensiones surgen de la necesidad del sector laboral



% % % % % % % % % % % % % % % % % % % % % % % % % % % % % % % % % % % % % % % %
\newpage

\section{Justificación}

La \textit{Licenciatura en Matemática Aplicada} de la \textit{Universidad de San Carlos de Guatemala} (USAC) tiene como fin primordial contribuir al desarrollo de la ciencia del país, a través de la formación de profesionales con un enfoque curricular y un modelo pedagógico que le capacitan para desempeñarse con éxito en los distintos sectores del campo laboral, de esta manera reduciendo la dependencia de científica de los países desarrollados.\bigskip

En Guatemala los planes de pensiones ocupacionales, en cuanto al análisis de su situación financiera, han dado mayor importancia al método del análisis del flujo de efectivo, mientras que las normas internacionales dictan que estos planes deben ser evaluados principalmente por el nivel de financiamiento de su reserva matemática.  \bigskip

El estudio de los aspectos matemáticos de ambos métodos presentados mediante este trabajo de graduación, determinará la conveniencia de su aplicación en los planes de pensiones ocupacionales. Dejando de esta forma, un presedente que sirva de base en el ámbito guatemalteco, para análisis posteriores de la situación financiera de diversos planes de pensiones ocupacionales.



% % % % % % % % % % % % % % % % % % % % % % % % % % % % % % % % % % % % % % % %
\newpage

\section{Marco teórico}

Los planes de pensiones ocupacionales, son beneficios que el sector laboral privado provee a sus trabajadores como un complemento de la seguridad social del país. Dichos planes, son evaluados períodicamente determinando así la situación financiera del mismo, para posteriormente determinar el valor de la prima en función de dicho período.

%Los \textit{métodos de valuación actuarial} para determinar la situación financiera de un plan de pensiones son los siguientes: \textbf{}

% % % % % % % % % % % % % % % % % % % % % % % % % % % % % % % % % % % % % % % %
\newpage

\section{Objetivos}


\subsection{Objetivo general}

\begin{itemize}
	\item Determinar los beneficios de evaluar la situación financiera de un plan de pensiones ocupacional, estimando el nivel del financiamiento de su reserva matemática contra el flujo de efectivo.
\end{itemize}


\subsection{Objetivos específicos}

\begin{enumerate}
	\item Identificar los principales beneficios de los planes ocupacionales.
	\item Identificar los elementos de la matemática actuarial necesarios para el cálculo de pensiones.
	\item Revisar los aspectos matemáticos de los métodos de valuación actuarial.
	\item Comprobar la conveniencia de la aplicación de los métodos de valuación en los planes de pensiones ocupacionales. 
\end{enumerate}

% % % % % % % % % % % % % % % % % % % % % % % % % % % % % % % % % % % % % % % %
\newpage

\section{Metodología}

La metodología a utilizar en el desarrollo de este trabajo de graduación, se llevará a cabo de la siguiente manera:

\begin{itemize}
	\item Se consultará, en primera instancia, diversos documentos de acceso público de la Organización Internacional del Trabajo, incluídas en la bibliografía preliminar,  en los cuales establecen las normas y convenios para la estructuración de los distintos planes de pensiones.
	
	\item 

	
	\item 
	
	\item 	
\end{itemize}

% % % % % % % % % % % % % % % % % % % % % % % % % % % % % % % % % % % % % % % %
\newpage

\section{Cronograma}
Se tabulan las actividades previstas en el desarrollo del trabajo de graduación.
% El cronograma es un cuadro semejante al utilizado en los anteproyectos FACYT,
% meses en las etiquetas de las columnas y actividades con etiquetas de las filas, 
% se sombrea (marca) la celda correspondiente.

\vspace{1cm}

\begin{tabular}{|c|c|c|}
  \hline
  % after \\: \hline or \cline{col1-col2} \cline{col3-col4} ...
   \textbf{No.}  &  \textbf{Actividad a realizar} &  \textbf{Duración }\\ \hline
  1  & Elaboración del protocolo & 2 semanas \\
  2  & Búsqueda de bibliografía & 3 semanas \\
  3  & Capítulo 1: Planes de Pensiones  & 5 semanas \\
  4  & Capítulo 2: Matemática Actuarial de los Planes de Pensiones & 6 semanas \\
  5  & Capítulo 3: Tipos de Métodos de Valuación & 5 semanas \\
  6  & Capítulo 4: Aplicación de los Métodos de Valuación en los Planes de Pensiones & 6 semanas \\
  7  & Escritura de informe final & 1 semana \\
  8  & Revisión del asesor de tesis & 1 semana \\
  9 & Revisión del revisor de Escuela de Ciencias & 1-2 semanas \\
  10 & Revisión departamento de Lingüística & 1-2 semanas \\
  11 & Trámites finales & 1 semana  \\
  12 & Impresión informe final & 2-3 días \\
  13 & Solicitud Examen Público & 1 semana \\
  \hline
\end{tabular}

% % % % % % % % % % % % % % % % % % % % % % % % % % % % % % % % % % % % % % % %
\newpage

\section{Índice preliminar}
El índice preliminar del trabajo de graduación es el siguiente
\begin{align*}
&\text{ÍNDICE DE ILUSTRACIONES}\\
&\text{LISTADO DE SÍMBOLOS}\\
&\text{GLOSARIO}\\
&\text{INTRODUCCIÓN}\\
&\text{1. PLANES DE PENSIONES}\\
&\text{1.1. ASPECTO HISTÓRICO }\\
&\text{1.2. CUIDADO MÉDICO }\\
&\text{1.3. BENEFICIOS DE VEJEZ}\\
&\text{1.4. SUPERVIVENCIA}\\
&\text{1.5. INVALIDEZ }\\
&\text{2. MATEMÁTICA ACTUARIAL DE LOS PLANES DE PENSIONES}\\
&\text{2.1. CÁLCULO DE LA PRIMA}\\
&\text{3. MÉTODOS DE VALUACIÓN}\\
&\text{3.1. EL FLUJO DE EFECTIVO}\\
&\text{3.2. EL FINANCIAMIENTO DE SU RESERVA MATEMÁTICA }\\
&\text{4. APLICACIÓN DE LOS MÉTODOS DE VALUACIÓN EN LOS PLANES DE PENSIONES}\\
&\text{BIBLIOGRAFÍA} % publicaciones que no se citaron pero sirvieron para la elaboración del trabajo
\end{align*}

% % % % % % % % % % % % % % % % % % % % % % % % % % % % % % % % % % % % % % % %
\newpage

%\section{Bibliografía preliminar}
\bibliography{bibliotesis}
\label{fin}
\end{document} 
