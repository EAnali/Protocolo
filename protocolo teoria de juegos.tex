%% Correcciones y sugerencias por William Gutiérrez
%% para no perder la costumbre :-)
%% Debe incluir al Dr. Sergio López, asesor principal
\documentclass[12pt,letterpaper,titlepage]{article}
\usepackage[spanish]{babel}
\usepackage[utf8x]{inputenc}
\usepackage{amsmath,amssymb}
\usepackage{graphicx}
\usepackage{amsthm}
\usepackage{latexsym}
\usepackage{cancel}
\usepackage{mathrsfs,amsfonts,mathptmx}
\usepackage[text={5.8in,8.6in},centering]{geometry}
\renewcommand{\spanishoperators}{sen spec d}
\renewcommand{\baselinestretch}{1.6}
\makeatletter \decimalpoint
  \def\th@exercise{%
    \normalfont % body font
    \thm@headpunct{:}%
  }

\makeatother
\usepackage{hyperref}
\hypersetup{% bookmarksnumbered,
  bookmarksopen,
  pdfpagelayout=OneColumn,
  pdfview=FitH,
  pdfstartview=FitH,
  pdfborder={0 0 0}}




% \theoremstyle{definition}
\renewcommand{\spanishrefname}{Bibliografía preliminar}

\title{Protocolo de trabajo de graduación:\\
Teoría de Juegos}
\author{Br. Emilene Analí Romero Marroquín\\Carné: 2012-13113\\Dirección: 7a. Ave. "A" 9-29 Zona 11,\\Mixco, Guatemala\\Teléfono: 3335\,4941\\Asesores: Lic. Hugo Allan García Monterrosa\\Licenciatura en Matemática Aplicada\\Universidad de San Carlos de Guatemala\\Número de páginas: \pageref{fin}}
\date{Guatemala, \today}

\begin{document}
%\begin{titlepage}
%\renewcommand{\thepage}{}
%\pagestyle{empty}
\maketitle
%\end{titlepage}\newpage
\setcounter{page}{2}
\tableofcontents

% % % % % % % % % % % % % % % % % % % % % % % % % % % % % % % % % % % % % % % %
\newpage

\section{Introducción}
% Para los pies de página, las definiciones deben ser coloquiales para no ser tan técnicos en la introducción.
La \textit{Teoría de Juegos} 
% % % % % % % % % % % % % % % % % % % % % % % % % % % % % % % % % % % % % % % %
\newpage

\section{Justificación}


% \footnote{Hay redundancia en esta parte, buscar otros palabras.}



% % % % % % % % % % % % % % % % % % % % % % % % % % % % % % % % % % % % % % % %
\newpage

\section{Marco teórico}


% % % % % % % % % % % % % % % % % % % % % % % % % % % % % % % % % % % % % % % %
\newpage

\section{Objetivos}


\subsection{Objetivo general}


\subsection{Objetivos específicos}

% % % % % % % % % % % % % % % % % % % % % % % % % % % % % % % % % % % % % % % %
\newpage

\section{Metodología}


% % % % % % % % % % % % % % % % % % % % % % % % % % % % % % % % % % % % % % % %
\newpage

\section{Cronograma}
Se tabula las actividades previstas en el desarrollo del trabajo de graduación.
% El cronograma es un cuadro semejante al utilizado en los anteproyectos FACYT,
% meses en las etiquetas de las columnas y actividades con etiquetas de las filas, 
% se sombrea (marca) la celda correspondiente.

\vspace{1cm}

\begin{tabular}{|c|c|c|}
  \hline
  % after \\: \hline or \cline{col1-col2} \cline{col3-col4} ...
   \textbf{No.}  &  \textbf{Actividad a realizar} &  \textbf{Duración }\\ \hline
  1  & Curso de propedéutico de trabajos de graduación & Concluido \\
  2  & Elaboración del protocolo & Concluido \\
  3  & Sección: Antecedentes & 1 semana \\
  4  & Sección: Grupo-anillos & 2 semanas \\
  5  & Sección: Teoría de representación de grupos & 2 semanas \\
  6  & Sección: Unidades de los grupo-anillos & 2 semanas \\
  7  & Sección: Aplicaciones & 1 semana \\
  8  & Escritura de informe final & 1 semana \\
  9  & Revisión del asesor de tesis & 1 semana \\
  10 & Revisión del revisor de Escuela de Ciencias & 1-2 semanas \\
  11 & Revisión departamento de Lingüística & 1-2 semanas \\
  12 & Trámites finales & 1 semana  \\
  13 & Impresión informe final & 2-3 días \\
  14 & Solicitud Examen Público & 1 semana \\
  \hline
\end{tabular}

% % % % % % % % % % % % % % % % % % % % % % % % % % % % % % % % % % % % % % % %
\newpage

\section{Índice preliminar}
El índice preliminar del trabajo de graduación es el siguiente
\begin{align*}
&\text{ÍNDICE DE ILUSTRACIONES}\\
&\text{LISTADO DE SÍMBOLOS}\\
&\text{GLOSARIO}\\
&\text{INTRODUCCIÓN}\\
&\text{1. TEORÍA PRELIMINAR}\\
&\qquad\text{1.1 HISTORIA}\\
&\qquad\text{1.2 TEORÍA DE PROBABILIDAD}\\
&\qquad\text{1.2.1}\\
&\qquad\text{1.3 TEORÍA DE LA UTILIDAD}\\
&\qquad\text{1.4 TEORÍA DE JUEGOS}\\
&\text{2. JUEGOS }\\
&\qquad\text{2.1 }\\
&\qquad\text{2.2 }\\
&\qquad\text{2.3 }\\
&\text{3. JUEGOS DINÁMICOS }\\
&\qquad\text{3.1 }\\
&\qquad\text{3.2 }\\
&\text{4. APLICACIONES }\\
&\qquad\text{4.1}\\
\end{align*}

\newpage
\begin{align*}
&\qquad\text{4.2 GRUPOS FINITOS}\\
&\text{5. APLICACIONES DE LOS GRUPO-ANILLOS }\\
&\text{CONCLUSIONES}\\
&\text{RECOMENDACIONES}\\ % piden esta onda
&\text{REFERENCIAS}\\ % publicaciones que se citan dentro del trabajo
&\text{BIBLIOGRAFÍA} % publicaciones que no se citaron pero sirvieron para la elaboración del trabajo
\end{align*}

% % % % % % % % % % % % % % % % % % % % % % % % % % % % % % % % % % % % % % % %
\newpage

%\section{Bibliografía preliminar}
\begin{thebibliography}{99}
	\addcontentsline{toc}{section}{Bibliografía preliminar}
% Nombres completos o iniciales / mayúsculas o minúsculas jojo jo
% faltan datos en algunas referencias, casa editorial por ejemplo
\bibitem{b1} C. Polcino Milies y K. Sehgal. \textbf{An Introduction to Group Rings.} EUA: Kluwer Academic Publishers (2002).
\bibitem{b2} Hernstein, I.N. \textbf{Topics in Algebra.} EUA: John Wiley and Sons (1966).
\bibitem{b3} Lang, Serge. \textbf{Linear Algebra.} (Undergraduate Texts in Mathematics) EUA: Springer (1966).
\bibitem{b4} McLane, S. \textbf{Categories for a working Mathematician.} 2"a ed. (Graduate Texts in Mathematics 5) EUA: Springer (1998).
\bibitem{b5} Beachy, John. \textbf{Introductory Lectures on Rings and Modules.} Reino Unido: Cambridge University Press edition (1999).
\bibitem{b6} Agoston, Max. \textbf{Algebraic topology: a first course.} EUA: Dekker, Marcel Inc. (1976)
\bibitem{b7} D.S. Passman. \textbf{The algebraic Structure of Group Rings.} EUA: Wiley-In"-terscience (1977).
\bibitem{b8} T. Molien. \textbf{Über die Invarianten der linearen Substituionsgruppen}, S'ber Akad. d. Wiss. Berlin (1897), 1152-1156.
\bibitem{b9} B. Hurley and T. Hurley. \textbf{Group ring cryptography}, \texttt{	arXiv:1104.1724v1 [math.GR] } (2011) .
\bibitem{b10} E. Noether. \textbf{Hyperconplexe Grössen und Darstellungtheorie}, Math. Z. 30 (1929), 641-692.
\bibitem{b11} R. Brauer and E. Noether. \textbf{Über minimale Zerfällungskörper irreduzibler Darstellungen}, Sitz. Preuss. Akad. Wiss. Berlin (1927), 221-228.
\bibitem{b12} R.J. McEliece. \textbf{The Algebraic Theory of Convolutional Codes}, The Handbook of Coding Theory, EUA (1983), 1:1065-1138.
% nueva bibliografía, para algunas referencias de fechas
\bibitem{b13} Stewart, Ian. \textbf{De aquí al infinito -- Las matemáticas de hoy.} España: Editorial Crítica (1998).
\bibitem{b14} Bell, E.T. \textbf{Los grandes matemáticos.} Argentina: Editorial Losada (1948).
\bibitem{herr} Gutiérrez, William. \textbf{Introducción a \TeX\ y a \LaTeXe.} Guatemala: Facultad de Ingeniería, USAC (2010).
\end{thebibliography}

\label{fin}
\end{document} 
